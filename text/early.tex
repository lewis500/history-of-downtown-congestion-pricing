\section{Early steps}

This section treats a period from the mid 1950's until the early 1970's, during which the idea of pricing urban areas was incubated. The section concludes by pointing out a family resemblance between how British scholars of that era conceived of traffic in towns and how they planned workable pricing schemes.

\citet{buchanan1952} makes passing mention of a ``city license,'' while \citet{Walters1954} explains in more detail: ``For example, a special `London licence' would have to be acquired and displayed before vehicles could use the roads of central London between 8 a.m. and 6 p.m.'' But the first detailed proposal for downtown pricing came in a 1959 testimony to the US Congress by the economist William Vickrey \citep{Vickrey1959}. Vickrey's testimony is, in part, a response to a difficulty posed by Dick Netzer of the Federal Reserve Bank of Chicago four years ealier:
\begin{quote}
 We cannot construct a tollgate at each and every intersection, and there is no way of metering highway service---like electric or water service, for example---that would take a meter on each vehicle which records only only the mileage but the kind of road used, the time of day, and so on. (\citet{Netzer1955} quoted in \citet{Vickrey1959})
\end{quote}

Vickrey's answer was wireless, electronic tolling that would obviate the need for congestion-inducing tollbooths. The Washington, D.C. metropolitan area would be divided into a large number of zones (``between 60 and 200''), and cars would carry transponders. Along the borders of the pricing zones, interrogators would signal the transponders on passing cars with electromagnetic waves, whereupon the transponders would return an identifying signal---essentially the concept behind RFID-based electronic tolling that is widespread today. Selective camera enforcement would do for cars without transponders. Remarkably, Vickrey not only supplied detailed costs and technical specifications for much of the system---from transmission to data storage---but actually demonstrated the transponder apparatus for Congress using a model train set-up.\footnote{Richard Arnott and Marvin Krauss later cleaned up Vickrey's testimony and published it as \citet{Vickrey1994}, but the author has obtained the original document and can supply it on request.} 

At the same time Vickrey was at work on Washington's problems, British officials perceived their own country to be approaching a traffic crisis: the number of registered vehicles had doubled from 4.5 million in 1950 to 9 million in 1960, but state expenditure on roads was low, and there was as yet no motorway network \citep[p.523-524]{Gunn2011}. While building motorways could be hoped to alleviate intercity congestion, urban traffic in Britain's old cities remained an unsolved and growing problem. Therefore, in 1960, the UK government commissioned the urban planner Colin Buchanan, who had written a book on cars in cities, to analyze the question of city traffic and propose planning remedies. 

Other British researchers, however, thought that road pricing held more promise than reconfiguring city streets and land-use---particularly for cities where the cost of construction was high and the extant road network ancient.\footnote{See \citep{Rooney2014} for an account of the ideological conflict between the planning and pricing approaches to managing urban traffic during this era.} Gabriel Roth, a researcher at Cambridge, asked, ``Before pulling down the fabric of the country let us at least consider the economic implications of what we are doing'' \citep[p.289]{Roth1961}. The same year, Walters published \citet{Walters1961}---one of the most cited works in transportation economics---which couched orthodox relations among flow and speed in terms of the supply-and-demand diagram of microeconomics. While focused on highways, the paper mentions pricing might be applied to city streets using a special mileometer, which---once the driver turned it on---would display a flag that roadside observers could check.

In July 1962, the Ministry of Transport convened a group of researchers to discuss road pricing, whereafter officials established a committee---helmed by Reuben Smeed of the UK's Road Research Laboratory and including Alan Walters and Gabriel Roth---to write a report on road pricing \citep{Rooney2014}. After a series of meetings, the report---commonly referred to as the ``Smeed Report''---was published in July 1964. The Smeed Report, \citet{MoT1964}, sets out principles for a good urban road pricing system and proposes a large number of schemes with varying levels of technical complexity. These schemes include simple parking surcharges, daily licenses (pieces of paper or stickers that must be displayed to enter certain zone), electronic systems like Vickrey's, and systems where drivers activate timers or mileometers when they travel through congested areas. For instance, one system DESCRIBE THE COLORED LIGHT SYSTEM.


Ultimately, the report was unpopular with the government. INCLUDE DETAILS LATER. OVERSHADOWED BY BUCHANAN.

Still, scholarly work on pricing continued apace (CITE SOME EXAMPLES). The World Bank commissioned a study by the US Consultancy Alan M. Vorhees for road pricing in Caracas, Venezuela \citep{Vorhees1973}, which would seem to have influenced Singapore's decision to adopt road pricing, discussed below.

% \subsection*{The idea of zones}

% TALK HERE ABOUT ALL THE ZONE SCHEMES. THEY DIDN"T SEEM TO ACIVELY ACKNOWLEDGE THE CRUCIAL DIFFERENCE BETWEEN A ZONE AND A LINK, BUT AT THE SAME TIME THEY WERE DOING RESEARCH ON ZONAL RELATIONS.
A characteristic common to many of the schemes is the division of cities into zones, where the zones with the highest demand for road travel would be the most expensive to traverse. 

One heretofore neglected aspect of this line of thought is that, at the same time British researchers were coming up with schemes to price cities by the zone, they were also developing a macroscopic 
