\section{First steps}\label{sec:early}

Economists and engineers have written about congestion pricing since at least Adam Smith \citep{Lindsey2006}. Early theoretical work---e.g., \citet{Pigou1920} and \citet{Beckmann1956}---is concerned with \emph{links} such as roads and bridges. But link-by-link tolling has been impractical in city centers, where roads intersect constantly, and so in the 1950's scholars began to consider the idea of pricing downtown zones as singular entities. The first DCP proposal seems to be \citet{buchanan1952}'s passing mention of a ``city license,'' which \citet{Walters1954} somewhat fleshes out: ``A special `London licence' would have to be acquired and displayed before vehicles could use the roads of central London between 8 a.m. and 6 p.m.'' But the first real proposal for DCP came in a 1959 testimony to the US Congress about traffic problems in Washington, D.C. by the economist William Vickrey \citep{Vickrey1959}.\footnote{ Richard Arnott and Marvin Krauss later cleaned up Vickrey's notes for his testimony and published the result as \citet{Vickrey1994}.}

Vickrey proposed that drivers pay a time-varying toll each time they entered one of dozens of zones throughout the the Washington, D.C. area. Enforcement would consist of what \citet{DePalma2011} call a ``tag and beacon'' system: tolling sites along the zone boundaries have a ``beacon'' located over, beside or under the roadway, which constantly emits electromagnetic waves (e.g., radio or microwave). These waves are picked up by ``tags'' (transponders) planted in vehicles which fire back an identifier code, leading to a charge. Vickrey provided detailed technical specifications and costs estimates for all parts of the scheme, and even demonstrated the transponder system for Congress using a model train, but nothing directly came of the proposal.

Meanwhile, British officials perceived an impending crisis: the number of registered vehicles had doubled in the 1950's, but road spending was low and British street networks ill-equipped to deal with a flood of automobiles \citep{Gunn2011,Hall2004}. While planned motorways would alleviate intercity congestion, traffic in city centers was an unresolved problem, and so the Ministry of Transport comissioned several reports tackling urban congestion. One, the ``Buchanan Report'' \citep{MoT1963}, written by planner Sir Colin Buchanan, recommended planning solutions such as road-building, pedestrianized streets, environmental standards and parking garages. The other, the ``Smeed Report'' \citep{MoT1964}, written by a panel of academic economists and researchers from the Road Research Laboratory under the leadership of Reuben Smeed, set out principles for DCP and considered variously complex schemes. These included wireless systems like Vickrey's, licenses like Walters' (who was on the panel) and metering systems using devices mounted on cars to track time and distance traveled. But in the end, it was the Buchanan Report that captured the government and public's attention.\footnote{\citet{Rooney2014} gives a detailed account of the story behind the two reports and the ideological conflict between proponents of each approach.}, receiving nationwide acclaim, although the government did set up a working group for further study of pricing and other measures.  That group's report, \citet{MoT1967}, concluded that pricing might be effective but was not yet practical and more research was needed.

While DCP would be delayed in Britain, members of the Smeed Report committee continued to work on the idea up to and after the first implementation of DCP in Singapore, described in the next section. J.M. Thomson conducted the first serious modelling study \citep{Thomson1967a}, of DCP for London---a topic of numerous and explicit government studies over the subsequent thirty years.\footnote{These studies are described in detail in \citet[Ch. 4]{Richards2006} and \citet[Ch. 4]{Gomez-Ibanez1994}.} Walters wrote a monograph, \citet{Walters1968}, about road pricing for the World Bank, who then funded a study, \citet{Vorhees1973}, on a supplementary licensing scheme for Caracas. While Venezuela declined the proposal, Gabriel Roth went to work for the World Bank as a consultant and wound up working on transit in Singapore at a fortuitous moment.
