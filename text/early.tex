\section{Early steps}

The idea of DRP was first incubated from the mid 1950's until the early 1970's. The first such proposal seems to be \citet{buchanan1952}'s passing mention of a ``city license,'' an idea \citet{Walters1954} fleshes out in somewhat more detail: ``A special `London licence' would have to be acquired and displayed before vehicles could use the roads of central London between 8 a.m. and 6 p.m.'' But the first detailed proposal came in a 1959 testimony to the US Congress about traffic problems in Washington, D.C. by the economist William Vickrey \citep{Vickrey1959}. 

Vickrey proposed a system whereby the the Washington, D.C. metropolitan area would be divided into zones (``between 60 and 200''). Drivers would be charged a time-varying toll whenever they crossed one of the zone boundaries. For enforcement, Vickrey proposed an instance of what \citet{DePalma2011} call a ``tag and beacon'' system, which is now most commonly used on toll roads and bridges. In such a system, toll sites have a ``beacon'' located over, beside or under the roadway, which constantly emits electromagnetic waves  (e.g., radio or microwave). Vehicles carry a ``tag'' which, when struck by these waves,  fires back an identifier code. A central system then tallies a charge against the account or vehicle-owner associated. Remarkably, Vickrey actually demonstrated the transponder apparatus for Congress using a model train .\footnote{Richard Arnott and Marvin Krauss later cleaned up Vickrey's testimony and published it as \citet{Vickrey1994}.} Nothing, however, directly came of Vickrey's scheme.

While Vickrey was at work on Washington's problems, British officials perceived a traffic crisis: the number of registered vehicles had doubled in the 1950's, but expenditure on roads was low and British towns had ancient street layouts \citep[p.523-524]{Gunn2011}. To deal with urban congestion, the Ministry of Transport comissioned several reports. One, the ``Buchanan Report'' (CITE) recommended planning solutions and city reconstruction. The other, the ``Smeed Report'' \citep{MoT1964}, set out principles and proposes schemes of varying levels of technical complexity. But ultimately, it was the Buchanan Report that wound up influencing the government more in the short term.\footnote{See \citep{Rooney2014} for an account of the conflict between the planning and pricing approaches.}

MAKE SOME CONCLUDING STATEMENT

