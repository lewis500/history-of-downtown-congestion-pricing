\section{Early steps}

The first DCP proposal seems to be \citet{buchanan1952}'s passing mention of a ``city license,'' which \citet{Walters1954} fleshes out in more detail: ``A special `London licence' would have to be acquired and displayed before vehicles could use the roads of central London between 8 a.m. and 6 p.m.'' But the first real proposal came in a 1959 testimony to the US Congress about traffic problems in Washington, D.C. by the economist William Vickrey \citep{Vickrey1959}. 

Vickrey proposed that drivers pay a time-varying toll each time they entered one of dozens of zones throughout the the Washington, D.C. area. Enforcement would consist of what \citet{DePalma2011} call a ``tag and beacon'' system: tolling sites along the zone boundaries have a ``beacon'' located over, beside or under the roadway, which constantly emits electromagnetic waves (e.g., radio or microwave). These waves are picked up by ``tags'' (transponders) planted in vehicles which fire back an identifier code, leading to a charge against the person associated with the tag.\footnote{Vickrey even demonstrated the transponder system on the floor of Congress using a model train, and built a similar system in his own driveway. Richard Arnott and Marvin Krauss later cleaned up Vickrey's testimony and published it as \citet{Vickrey1994}.} Nothing, however, directly came of the proposal.

Meanwhile, British officials perceived an impending traffic crisis: the number of registered vehicles had doubled in the 1950's, but road spending was low and there was only one motorway \citep{Gunn2011,Hall2004}. Traffic in city centers was of particular concern given the countries' old street network, and so the Ministry of Transport comissioned several reports tackling urban congestion. One, the ``Buchanan Report'' \citep{MoT1963}, written by planner Sir Colin Buchanan, recommended planning solutions such as road-building, pedestrianized streets, environmental standards and parking garages. The other, the ``Smeed Report'' \citep{MoT1964}, written by a panel of academic economists and researchers from the Road Research Laboratory, set out principles for DCP and considered variously complex schemes. These included wireless systems like Vickrey's, ``supplementary licenses'' like Walters' (who was on the panel) and simple parking charges. In the end, it was the Buchanan Report that captured the government and public's attention, receiving nationwide acclaim, although the government did set up a working group for further study of pricing and other measures \citep{Rooney2014}.\footnote{XXX DISCUSS ROONEY} The group's report, \citet{MoT1967}, concluded that pricing might be effective but was not clearly practical in the short run, leading the government to postpone the matter. 

Although decades would pass before a British city adopted road pricing, today it is obvious that this early era laid the intellectual groundwork for implementation in the UK and elsewhere. While working on \citet{MoT1967}, J.M. Thomson (who had helped with the Smeed Report) conducted the first serious modelling study of supplementary licensing for London---\citet{Thomson1967a}---which considered a system very similar to the one later adopted. This commenced a long series of studies on DCP in London. Walters wrote a monograph, \citet{Walters1968}, about road pricing for the World Bank, who then funded a study, \citet{Vorhees1973}, on a supplementary licensing scheme for Caracas which---while never implemented---seems to have helped inspire the Singapore scheme discussed next.

