\section{A transition period}\label{sec:transition}

After ALS, no new DCP schemes were launched until 1998. Still, there were constant developments that make the interval 1975-1998 a ``transition period'' (equivalent to the High Middle Ages in Europe). During this time governments and researchers carried out studies of detailed proposals and demonstrated new technology which laid the groundwork for a renaissance in pricing later.

\subsection{London}



\subsection{Norwegian toll rings}

Around 1990, the three largest Norwegian cities implemented ``toll rings,'' in which a cordon of tolling sites encloses an entire city. There are not DCP as we have defined it; they are pure revenue devices with little intended or realized effect on traffic. But they demonstrated new technologies as well as a new business model: governments in some region draw up a list of transportation projects, then incorporate a toll company to pay for them \citep{Ieromonachou2006}. 

Bergen launched the first toll ring in January 1986 with six tolling sites on arterials leading into the city \citep{Ieromonachou2006,Ramjerdi2004}. At each site, half the lanes had staffed toll booths and the other half were non-stop lanes reserved for cars displaying ``season pass'' stickers. Oslo opened its toll ring in January 1990 using a microwave tag-and-beacon system alongside tollbooths and coin payment lanes. In October 1991, Trondheim opened the third toll ring using the same electronic system as Oslo, but with tolls varying slightly over the day.

 \subsection{Stockholm}

Discussion of DCP in Stockholm started in the 1980s, with a proposal that cars have to display a transit pass in their windshields \citep{GullbergIsaksson2009,Arnott2005}. In 1989, the City of Stockholm formulated plans for this ``car card'' proposal as well as an electronic cordon pricing system \citep[p. 90]{Hau1992}; but since road pricing was deemed a ``tax'' rather than a ``charge,'' Swedish law required that the national parliament approve it. Despite Stockholm's support for the idea, parliament tabled the tax for complex political reasons \citep{Ahlstrand2001}. 

Meanwhile, the national government convened Stockholm's major political parties to negotiate an infrastructure package for the region. The resulting agreement, finalized in 1992 and called the ``Dennis Agreement'' after the official tasked with mediating the negotiations, resembled the Norwegian toll ring model: tolls in a cordon around central Stockholm would pay for massive highway and tunnel projects intended to deflect traffic  (see \citet[pp. 39-40]{Gomez-Ibanez1994} and \citet[p. 92]{Hau1992}). However, in 1997  the Dennis Agreement fell apart due, as before, to political maneuvers \citep{Ahlstrand2001,GullbergIsaksson2009}.

\subsection{Sidelined proposals}

The Norwegian toll rings were actually implemented, and Stockholm and London would adopt schemes after the turn of the century, but during this period there were also several schemes proposed and studied but never implemented.

 Between 1983 and 1985, Hong Kong conducted studies of a DCP system called ``Electronic Road Pricing'' (ERP). ERP was to be a tag-and-beacon system, wherein inductive loops embedded in the pavement at tolling sites---positioned to form cordons---would harvest ID's from transponders placed underneath vehicles \citep{Dawson1986}. Users received a monthly bill listing the price, place and times of all charging events. Field trials involving 2,500 equipped vehicles and 18 tolling sites showed the technology to be reliable. Modelling studies considered three scenarios for the charging structure and topology, wherein Hong Kong would be divided into up to 13 zones with prices that varied by time-of-day and direction of travel \citep[Table 11, p. 23]{Gomez-Ibanez1994}.  Despite succesful trials and positive forecasts from desktop modelling, Hong Kong did not adopt ERP when the study concluded. Many factors contributed \citep{Hau1990,Borins1988}, but two stand out. First, the trials ended exactly as outside developments (recession, new infrastructure, vehicle taxes) had mitigated congestion. Second, Britain was then negotiating to deliver Hong Kong to China in 1998, prompting worries about spying, and  ERP's monthly list of vehicle movements inflamed those concerns. While not implemented, the Hong Kong experience is important for demonstrating the technical feasibility of tag-and-beacon charging in a dense city, and its political failure influenced Singapore's choice of technology for its own ERP scheme.

