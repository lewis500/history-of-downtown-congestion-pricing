\section{A transition period: 1982-2003}

After ALS, no new DRP systems were launched until 1997. Still, there were two notable technological developments in the intervening years.

\subsection{Hong Kong Electronic Road Pricing}

 Between July 1983 and March 1985, consultants working for the government of Hong Kong conducted a study of a DRP system called ``Electronic Road Pricing'' (ERP), similar to the scheme Vickrey had proposed. The study involved two main avenues: modelling to compare system designs and field trials of the technology involving 2,500 equipped vehicles and 18 tolling sites\citep{Dawson1986}. Each vehicle had a solid-state device, called an Electronic Number Plate (ENP), placed underneath. When the ENP passed over inductive inductive loops embedded in the pavement, it transmitted an identifier linked to the vehicle's account \citep[pp. 130-131]{Dawson1986}. Users received a monthly bill (like a long-distance telephone bill) listing the time, place and times of all charging events. The modelling studies, meanwhile, considered three main scenarios for the charging structure and topology (see \citet[Table 11, p. 23]{Gomez-Ibanez1994} or \citet[Table 10.3, p. 218]{Small1998} for a summary). Tolling sites would be placed so as to divide Hong Kong into up to 13 zones with prices that varied by time-of-day and direction of travel.
 
 Despite succesful tests and modelling that forecast major gains, Hong Kong did not adopt ERP when the study concluded. Many factors contributed to the failure \citep{Hau1990,Borins1988}, but two seem to stand out. First, the trials ended exactly as outside developments (recession, new infrastructure, vehicle taxes) had mitigated congestion. Second, in December 1984, Britain had agreed to deliver Hong Kong to China in 1998, prompting worries about spying, and ERP's monthly bill listing vehicle movements inflamed such concerns.

\subsection{Norwegian Toll Rings}

Around 1990, the three largest Norwegian cities adopted so-called ``toll rings.'' These are not examples of DRP as we have defined it; they are pure revenue devices. As \citet{Ieromonachou2006} explains, in the toll ring model, local governments in some region draw up a list of transportation projects, then incorporate as shareholders of a toll company whos pay for these projects.
 
 Bergen launched the first toll ring in January 1986 with six tolling sites on arterials leading into the city \citep{Ieromonachou2006,Ramjerdi2004,Gomez-Ibanez1994}. At each site, half the lanes had staffed toll booths and the other half were non-stop lanes reserved for cars displaying ``season pass'' stickers---enforced by human inspection of video footage. Oslo opened its own toll ring in January 1990, to fund a package largely composed of tunnels intended to divert traffic off of city streets. Notably, Oslo's 19 tolling sites offered electronic collection using microwave communication, in addition to staffed tollbooths and coin payment lanes. Lastly, in October 1991, Trondheim opened the third toll ring using the same electronic system as Oslo. Trondheim's innovation was to slightly vary the fee over the day and charge for each crossing, whereas the other two charged the same charge all day and had offered passes granting unlimited entry. While none had much traffic impacts, the Norwegian toll rings are historically important for showcasing DRP's potential as a revenue source.

