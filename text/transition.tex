\section{A transition period: 1982-2003}

After ALS, no new downtown congestion pricing systems were launched until Singapore replaced ALS with an electronic system in 1997. Still, the era lasting from 1982 to 1997 included saw developments that laid the groundwork for wave of activity thereafter. This section describes several such developments: Hong Kong's Electronic Road Pricing trial, the Norwegian toll rings and field trials of a metering system in Cambridge. Additionally, during this era, London and Stockholm witnessed plans and political moves toward pricing that later bore fruit, but those events appear later in the schemes' designated sections.

\subsection{Hong Kong Electronic Road Pricing (ERP)}

Like Singapore, Hong Kong in the late 1970's struggled with the traffic consequences of a booming economy and scarcity of road space \citep{Hau1992}. Therefore, in 1979, the government announced its plans for transportation in a White Paper \citep{EnvironmentalBranch1979}, whose recommendations stemmed from the city's First Comprehensive Transport Study (Wilber Smith and Associates, 1976). The White Paper called for new expressways and rail transit, complemented by higher gas and vehicle ownership taxes (MAKE SURE THE WHITE PAPE DID ALL THIS).

The taxes' severity and bluntness invited interest in road pricing \citep[pp. 22-25]{Gomez-Ibanez1994}. In March 1983, the government contracted with consultants to design and test an electronic scheme. Their study lasted from July 1983 to March 1985 and involved two main avenues: modelling to compare system designs and field trials of the technology involving 2,500 equipped vehicles and 18 tolling sites\citep{Dawson1986}. 

The proposed system---called Electronic Road Pricing (ERP)---resembled the \citet{Vickrey1959} idea: a transponder called an Electronic Number Plate (ENP) was attached under the vehicle; when the vehicle passed over an inductive loop in the pavement, the ENP became energized and returned a identifier key to the loop, which passed the key to a roadside computer, which transmitted it via model to a central control center. \citep[pp. 130-131]{Dawson1986}. Users received a monthly bill (like a long-distance telephone bill) listing the time, place and times of all cordon traversals. Cameras at tolling sites would record plate numbers to catch violaters. The modelling studies considered three main scenarios for the charging structure and cordon topology (see \citet[Table 11, p. 23]{Gomez-Ibanez1994} or \citet[Table 10.3, p. 218]{Small1998} for a summary). Tolling sites would be placed so as to divide Hong Kong into up to 13 zones with prices that varied by time-of-day and direction of travel. 

Despite succesful tests and modelling that forecast major gains, Hong Kong did not adopt ERP when study concluded in 1985. Many factors contributed to the failure \citep{Hau1990,Borins1988}, but two seem to stand out. First, the trials ended exactly as outside developments (a recession, new infrastructure, the fiscal measures) had mitigated congestion. Second, in December 1984, Britain had agreed to deliver Hong Kong to China in 1998, prompting worries about spying, and ERP's monthly bill listing vehicle movements inflamed such concerns.

\subsection{Norwegian toll rings}

Around 1990, several Norwegian cities turned to so-called ``toll rings,'' which charge for entry to a central area. The toll rings were built to raise revenue and had little intended or realized impact on traffic, as their prices were too low. But they historically important, as the Oslo and Trondheim systems represent the first use of RFID to toll an entire urban area.

Bergen, Norway's second largest city, launched the first toll ring in January 1986 \citep{Ieromonachou2006}. The motivation was to provide funding for a master plan of new road projects. Six tolling sites were located on arterials leading to the urban area. Until electronic charging begain in 2004, half the lanes allowed payment at staffed toll booths, and the other half were non-stop lanes reserved for cars displaying ``season pass'' stickers that provided unlimited entries for a certain number of months \citep{Ramjerdi2004}. The season pass lanes were enforced by human inspection of footage from video cameras---not ANPR. Though technologically unsophisticated, the novel financing model of the Bergen ring proved popular.

Oslo, the largest city and capital of Norway, established its own toll ring in January 1990. The intent was to fund a very large package of road projects called the ``Oslo Package'' (since referred to as ``Oslo Package I,'' in light of its sequels), which was composed largely of tunnels designed to divert traffic off of city streets. WHAT ABOUT THE OTHER ONE?

\subsection{London studies}

Beginning in the 1960's, downtown pricing in London was the focus of constant study. \citet{Thomson1967a} considered a \pounds XXX daily license using stickers required for travel between 8AM and 7PM in an area of central London north of the Thames, but the national government decided to defer the matter. In the 1970's, the Greater London Council---a governing body over London's many small boroughs founded in 1968---commissioned research on a supplementary license for a larger zone that included all the land inside the London Inner Ring Road (including land south of the Thames) \citep{may1975}, which is the area London wound up using for the London Congestion Charge in 2003. But though research predicted large benefits, the Greater London Council---a governing body over all the municipalities of London founded in 1968---rejected the particular scheme design, citing concerns with equity and enforcement; instead they opted for parking controls \citep{Richards2006}. In 1986, due to the antogonism of the Council's leftist leader, Ken Livingston, the Conservative national government abolished the Greater London Council and replaced it with a less powerful body called the London Planning Advisory Committee (LPAC). LPAC endorsed the idea of downtown pricing early and then comissioned several very detailed and advanced studies on different scheme designs, which are described in \citet[p. 51-54]{Gomez-Ibanez1994}. An important finding of LPAC's research was that the public's response to pricing was much more positive when revenues were hypothecated to public transit rather than general government \citet[p. 51]{Richards2006}. Afterward, from 1991 to 1994, the UK Department of Transport launched the London Congestion Charging Research Programme, a series of still more detailed studies \citep{MVA1995,Richards1996}. But when the LCCRP's results were published in 1995, the Secretary of State for Transport decided that the technological problems loomed too large to implement downtown pricing for the time being.

 \subsection{Stockholm ``car card'' and Dennis Agreement}

Discussion of downtown pricing in Stockholm started in the early 1980s, when the local Social Democrats party proposed that cars entering downtown Stockholm be required to display a monthly pass---good for either transit or driving---in their windshields \citep{GullbergIsaksson2009,Arnott2005}. In 1989, the City of Stockholm formulated plans for this so-called ``car card'' proposal as well as an electronic cordon pricing system \citet[p. 90]{Hau1992} for details; but since road pricing was deemed to be a ``tax'' rather than a ``charge,'' Swedish law required that the national parliament approve the scheme. In Autumn 1990, although the local Social Democrats had proposed it, the Social Democrats who controlled the parliament wound up blocking pricing for political reasons \citep{Ahlstrand2001}. 

Meanwhile, recognizing that congestion in Stockholm had become serious, the national government convened Stockholm's political parties to negotiate an infrastructure package. The resulting agreement, finalized in 1992 and called the ``Dennis Package,''  consisted of \$6.1 billion in infrastructure projects designed to keep cars out of downtown Stockholm (see \citet[pp. 39-40]{Gomez-Ibanez1994} and \citet[p. 92]{Hau1992}). About 45\% of the money was slated for public transport, and the rest for large road projects: a bypass about 10km West of the City and a ring road around central Stockholm. To pay for the scheme, electronic tolls would be placed on the western bypass and along the cordon outside the ring road, so as to charge all traffic into Stockholm. However, in 1997 government cancelled the Dennis Package due, as before, to complex political maneuverings \citep{Ahlstrand2001,GullbergIsaksson2009}. Still, despite this failure, the Dennis Package had introduced a very specific road pricing plan into road pricing into Swedish politics and gained the attention of the country's environmental movement \citep[p.3]{Eliasson2014b}.

