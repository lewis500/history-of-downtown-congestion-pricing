\section{A transition period}\label{sec:transition}

The twenty years following the launch of ALS in 1975, can be considered a ``transition period'' (analagous to the High Middle Ages in Europe). No new schemes were implemented, but during this time governments crafted and studied DCP proposals and demonstrated new technologies, thereby laying the groundwork for a renaissance beginning in the late 1990's.

\subsection{London}

Beginning with \citet{Thomson1967a}, DCP in London became the focus of constant study. Thomson had considered a daily license using stickers required for travel between 8AM and 7PM in an area of central London north of the Thames. In the 1970's, the Greater London Council---a governing body over London's many small boroughs founded in 1968---commissioned research on a supplementary license for a larger zone that included all the land inside the London Inner Ring Road (including land south of the Thames) \citep{may1975}, which is the area London wound up using for the London Congestion Charge in 2003. But the Council rejected the scheme over equity and enforcement concerns \citep[Ch. XXX]{Richards2006}. In 1986, the Greater London Council was replaced with a less powerful body called the London Planning Advisory Committee (LPAC), which comissioned several very detailed and advanced studies (see summaries in \citet[p. 51-54]{Gomez-Ibanez1994}). Afterward, from 1991 to 1994, the Department of Transport launched the London Congestion Charging Research Programme (LCCRP), a series of still more detailed studies \citep{MVA1995,Richards1996}. But at LCCRP's conclusion the Secretary of State for Transport decided that technological impediments still loomed too large for implementation. 

\subsection{Hong Kong Electronic Road Pricing (ERP)}

 Between 1983 and 1985, Hong Kong conducted studies of a system called ``Electronic Road Pricing'' (ERP). ERP was to be a tag-and-beacon system, wherein inductive loops embedded in the pavement at tolling sites---positioned to form cordons---would harvest ID's from transponders placed underneath vehicles \citep{Dawson1986}. Users received a monthly bill listing the price, place and times of all charging events. Field trials involving 2,500 equipped vehicles and 18 tolling sites showed the technology to be reliable. Modelling studies considered three scenarios for the charging structure and topology, wherein Hong Kong would be divided into up to 13 zones with prices that varied by time-of-day and direction of travel \citep[Table 11, p. 23]{Gomez-Ibanez1994}.  Despite succesful trials and positive forecasts from desktop modelling, Hong Kong did not adopt ERP when the study concluded. Many factors contributed \citep{Hau1990,Borins1988}, but two stand out. First, the trials ended exactly as outside developments (recession, new infrastructure, vehicle taxes) had mitigated congestion. Second, Britain was then negotiating to deliver Hong Kong to China in 1998, prompting worries about spying, and  ERP's monthly list of vehicle movements inflamed those concerns. While not implemented, the Hong Kong experience is important for demonstrating the technical feasibility of tag-and-beacon charging in a dense city, and its political failure influenced Singapore's own ERP design, discussed later.

\subsection{Norwegian toll rings}

Around 1990, the three largest Norwegian cities implemented ``toll rings,'' in which a cordon of tolling sites encloses an entire city \citep{Ieromonachou2006,Ramjerdi2004}. 
Bergen launched the first toll ring in January 1986 with six tolling sites on arterials leading into the city . At each site, half the lanes had staffed toll booths and the other half were non-stop lanes reserved for cars displaying ``season pass'' stickers. Oslo opened its toll ring in January 1990 using a microwave tag-and-beacon system alongside tollbooths and coin payment lanes. In October 1991, Trondheim opened the third toll ring using the same electronic system as Oslo, but with tolls varying slightly over the day. None of the schemes are DCP as we have defined it; they are pure revenue devices with little intended or realized effect on traffic. But they demonstrated the feasibility of electronic tolling in an urban environment as well as a novel business model: governments in some region draw up a list of transportation projects, then implement tolls to pay for those particular projects. Later, Stockholm and Gothenburg used a similar financing model.

 \subsection{Stockholm}

Discussion of DCP in Stockholm started in the 1980s, with a proposal that cars have to display a transit pass in their windshields \citep{GullbergIsaksson2009,Arnott2005}. In 1989, the City of Stockholm formulated plans for this ``car card'' proposal as well as an electronic cordon pricing system \citep[p. 90]{Hau1992}; but since road pricing was deemed a ``tax'' rather than a ``charge,'' Swedish law required that the national parliament approve it. Despite Stockholm's support for the idea, parliament tabled the tax for complex political reasons \citep{Ahlstrand2001}. 

Meanwhile, the national government convened Stockholm's major political parties to negotiate an infrastructure package for the region. The resulting agreement, finalized in 1992 and called the ``Dennis Agreement'' after the official tasked with mediating the negotiations, resembled the Norwegian toll ring model: tolls in a cordon around central Stockholm would pay for massive highway and tunnel projects intended to deflect traffic  (see \citet[pp. 39-40]{Gomez-Ibanez1994} and \citet[p. 92]{Hau1992}). However, in 1997  the Dennis Agreement fell apart due, as before, to political maneuvers \citep{Ahlstrand2001,GullbergIsaksson2009}.

\subsection{Cambridge}

In 1990 the Cambridgeshire County Council gave preliminary approval to studying road pricing---partly as a way to pay for a light rail line \citep{Ison1996}. The system envisioned was called ``congestion metering.'' Every vehicle near Cambridge would have an in-vehicle unit (IU) connected to its odometer. The IU would have a mouth for a ``smart card'' that drivers could top up with value---like the RFID touch cards now used for transit. Inside the proposed charging zone, the IU would charge the smart card if odometer readings suggested the car had become stuck in traffic---e.g., if it took three minutes to move a half kilometer. The idea was to charge drivers only when they are experiencing, and hence causing, congestion. In October 1993 field trials, equipped cars drove between two beacons along a single road, and results suggested the technology was feasible. But the retirement of the scheme's chief advocate, a change of local government and negative public opinion ended the effort .




