\section{A transition period}

After ALS, no new DCP systems were launched until 1998, but there were several technological developments in the intervening years that paved the way for future schemes. This section describes two---Hong Kong's Electronic Road Pricing and the Norwegian toll rings---because of their scale and direct influence. But there were also  proposals drafted for DCP in other cities which never left the drawing board; see \citet{Gomez-Ibanez1994} and \citet{Small1998} for descriptions.

\subsection{Hong Kong Electronic Road Pricing}

 Between 1983 and 1985, Hong Kong conducted studies of a DCP system called ``Electronic Road Pricing'' (ERP). ERP was to be a tag-and-beacon system, wherein inductive loops embedded in the pavement at tolling sites---positioned to form cordons---would harvest ID's from transponders placed underneath vehicles \citep{Dawson1986}. Users received a monthly bill listing the price, place and times of all charging events. Field trials involving 2,500 equipped vehicles and 18 tolling sites showed the technology to be reliable. Meanwhile, modelling studies considered three main scenarios for the charging structure and topology \citep[Table 11, p. 23]{Gomez-Ibanez1994}. Tolling sites would be placed so as to divide Hong Kong into up to 13 zones with prices that varied by time-of-day and direction of travel. 

 Despite succesful tests and modelling that forecast large gains, Hong Kong did not adopt ERP when the study concluded. Many factors contributed \citep{Hau1990,Borins1988}, but two stand out. First, the trials ended exactly as outside developments (recession, new infrastructure, vehicle taxes) had mitigated congestion. Second, Britain was then negotiating to deliver Hong Kong to China in 1998, prompting worries about spying, and  ERP's monthly list of vehicle movements inflamed those concerns. Nontheless, although it was not implemented, the Hong Kong experience demonstrated the technical feasibility of wireless charging in a dense city, and its political failure influenced Singapore's choice of technology for its own ERP scheme.
 
\subsection{Norwegian Toll Rings}

Around 1990, the three largest Norwegian cities implemented ``toll rings,'' in which a cordon of tolling sites encloses an entire city. Bergen launched the first toll ring in January 1986 with six tolling sites on arterials leading into the city \citep{Ieromonachou2006,Ramjerdi2004,Gomez-Ibanez1994}. At each site, half the lanes had staffed toll booths and the other half were non-stop lanes reserved for cars displaying ``season pass'' stickers. Oslo opened its toll ring in January 1990 using a microwave tag-and-beacon system alongside tollbooths and coin payment lanes. In October 1991, Trondheim opened the third toll ring using the same electronic system as Oslo, but with tolls varying slightly over the day.

The toll rings are not examples of DRP as we have defined it; they are pure revenue devices with little intended or realized effect on traffic, due to their low prices and large cordons. What is novel about the toll rings is their business model:  governments in some region draw up a list of transportation projects, then incorporate a toll company whose revenues partly pay for these projects \citep{Ieromonachou2006}. Later, a similar model would motivate the adoption of the Stockholm and Gothenburg Congestion Taxes in nearby Sweden.
