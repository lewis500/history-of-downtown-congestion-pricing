\section{Singapore Area License Scheme}

\subsection{History}

In the early 1970's, Singapore took an interest in downtown pricing for the same reason Britain had ten years earlier: quick growth in car ownership. Between 1960 and 1970, the private vehicle population doubled as the country grew wealthier and built residential estates in its hinterlands, but the length of public roads rose only 35\% \citep[p.211-212]{Santos2004}. The government responded forcefully by consolidating bus operators, a six-fold increase in road spending from 1975 to 1980, plans for a metro and high taxes on car ownership\footnote{See Appendix B of \citet{Gomez-Ibanez1994} for a list of Singapore's policies against car ownership.} \citep{Santos2004}. 

In 1973, the government convened the Road Transport Action Committee (RTAC) to tackle downtown congestion. The Committee's first proposals included bus lanes as well as efforts to encourage staggered work hours (especially for public employees) and carpooling \citep{Chin1998}. In May 1974 the RTAC published a report declaring, ``Singapore cannot afford to continue to allocate scarce and valuable land to build unlimited miles of roads'' \citep[p.3]{SRTAC1974}. The plan proposed four policies: (i) sharp increases in CBD parking rates; (ii) new commuter bus services; (iii) a park-and-ride scheme whereby commuters would park in garages outside the CBD and take special shuttles to their workplaces; (iv) the Area License Scheme---a simple supplementary license system that made up the core of the plan.\footnote{It is possible the idea for ALS came when Gabriel Roth (of the Smeed committee), who was then working in Singapore as a consultant for the World Bank, passed the World Bank's plan for road pricing in Caracas, \citet{Vorhees1973}, to members of the RTAC (CITE ROTH 1996).} ALS commenced on June 3, 1975 and continued until 1998. 

\subsection{Design}

At shops and roadside booths, drivers paid S\$3 to buy a daily ``license'', or S\$60 for a monthly one, that permitted them to enter to a 6.2 km$^{2}$ Restricted Zone (RZ), Monday through Saturday morning \citep{WatsonHolland1978}. The ``license'' was a paper decal that went in the windshield. Wardens standing by the road at 22 access points to the RZ wrote down the plate numbers of vehicles lacking licenses. At first, only private vehicles with fewer than three passengers had to show licenses. Taxis, commercial goods vehicles, public vehicles, motorcycles, carpools and buses were exempt, but taxis lost their exemption within three weeks. While the charging period was 7:30-9:30 AM originally, authorities  extended charging to 10:15 AM on August 1. The Park-and-Ride scheme was essentially shut down within a few months for lack of ridership, and the new bus services did not draw many customers. 

There were many changes to ALS over the years.\footnote{See Table 1 of \citet[p. 98]{PhangToh1997} for a table listing changes. } In 1976, the price of a license was increased to S\$4 and a double rate charged to registered company cars, because firms could deduct the charge from taxes. In 1977, charges for taxis were cut to S\$2. In 1980, the license price rose to S\$5 (S\$10 for company cars). Throughout the 1980s, the boundaries of the RZ were expanded to enclose new developments. In 1989, carpools, goods vehicles and motorcycles lost their exempt status, and an evening charging period was added.\footnote{Note that the evening period was not an outbound charge, and a single license purchased travel in both periods.} In January 1994, a S\$2 license was added for entry from 10:15 AM to 4:30 PM \citep{PhangToh2004}. A vehicle entering the RZ in either or both peaks needed a Whole Day license (also good between the peaks), while a driver who entered only between the peaks only needed the Part Day license. 

Finally, while not technically part of ALS, in June 1995 the government implemented a parallel system called the Road Pricing Scheme (RPS) on one expressway. Drivers without a valid ALS license were required to buy a cheaper RPS license to drive there from 7:30-8:30 AM.

\subsection{Results}

Between March and October of 1975, vehicle entries to the RZ during the 7:30-10:15AM charging interval fell by 44\%---well beyond RTAC's desired 25-30\% reduction \citep{WatsonHolland1978}. Since the result exceeded expectations so drastically, some observers---including \citet{Wilson1988a,McCarthyTay1993} and \citet{WatsonHolland1978}---have concluded that initial charges were too high.

Speed results appear in Table \ref{tab:speed-singapore}. Commute trips into the RZ primarily switched to bus and carpool, while travelers who had traversed the RZ in the morning to destinations outside tended to switch to a ring road. 

A household survey of travel times conducted yielded disappointing results. There was almost no effect on traffic in the charge-free evening peak; and, due to the fall in speeds on the ring-road and mode shifting to slower modes such as bus and carpool, average journey times worsened in the short term. 

\begin{table}[ht]

\begin{tabular}{c>{\centering}p{3cm}>{\centering}p{3cm}}
 & before ALS\\
(kph estimated) & after ALS \\
(kph observed)\tabularnewline
\cline{2-3} 
Restricted Zone & 27 & 33\tabularnewline
\cline{2-3} 
inbound radials & 29 & 32\tabularnewline
\cline{2-3} 
outbound radials & 35 & 35\tabularnewline
\cline{2-3} 
ring road & 25 & 20\tabularnewline
\end{tabular}

\caption{Singapore speeds before and after Area License Scheme implementation \citep[p.10]{WatsonHolland1978} }
\label{tab:speed-singapore}
\end{table}

The 1989 reforms yielded major rebalancing of morning traffic flows: a rise and entries by car and taxi partly offset a roughly 50\% falls in entries by truck and motorcycle. In the new evening charging period, entries fell by 54\% and exits by 34\% \citep[p. 19]{Gomez-Ibanez1994}. Also, although less data are available for the 1994 introduction of the Part-Day pricing, traffic and congestion fell in the periods just after the morning and just before the evening period (CITE Phang et al., 1997).

\subsection{Finances}

ALS could be said to have the highest rate-of-return of any downtown pricing scheme. While the capital cost of implementation was S\$6.6 million, in fact 95 percent of that cost was sunk into the park-and-ride system \citet[p. 38]{WatsonHolland1978}. Initial revenues from license sales were S\$225,000 per month and operating costs were S\$50,000 per month. By 1993, annual revenues were S\$47 million, of which operating costs consumed 9 percent \citep{PhangToh2004}. In 1998, (just before the scheme ended) ALS and RPS earned about S\$100 million \citep{Chin2010}.