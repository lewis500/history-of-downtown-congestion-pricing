\section{Singapore Area License Scheme}

\subsection{History}

The motive for Singapore's interest in road pricing was similar to that of Britain's: explosive growth in car ownership. Between 1960 and 1970, the private vehicle population of Singapore doubled as the country grew wealthier and began to develop large residential estates in the island's hinterlands, but the total length of public roads rose only 35\% \citep[p.211-212]{Santos2004}. In response, throughout the 1970's the government undertook a menu of forceful policies which included: consolidating myriad bus operators into a single national company, a six-fold increase in road spending from 1975 to 1980, and high, escalating taxes on car purchases\footnote{See Appendix B of \citet{Gomez-Ibanez1994} for a list of  policies against car ownership.} \citep{Santos2004}. 

In 1973, the Singapore government convened the Road Transport Action Committee (RTAC) to propose solutions to congestion. The Committee's first proposals included bus lanes as well as efforts to encourage staggered work hours (especially for public employees) and carpooling \citep{Chin1998}. In May 1974 the RTAC published a report, \emph{A Plan for the Relief of Traffic Congestion in the City}, declaring, ``Singapore cannot afford to continue to allocate scarce and valuable land to build unlimited miles of roads to keep pace with this uncontrolled increase in traffic'' \citep[p.3]{SRTAC1974}. The plan proposed four policies: (i) sharp increases in CBD parking rates; (ii) new commuter bus services; (iii) a park-and-ride scheme whereby commuters would park in garages outside the CBD and take special shuttles to their workplaces; (iv) the Area License Scheme---a simple supplementary license system that made up the core of the plan.\footnote{It is possible the idea for ALS came when Gabriel Roth (of the Smeed committee), who was then working in Singapore as a consultant for the World Bank, passed the World Bank's plan for road pricing in Caracas, \citet{Vorhees1973}, to members of the RTAC (CITE ROTH 1996).} The Area License Scheme commenced operation on June 3, 1975 and continued until 1997, when an electronic system replaced it.

\subsection{Design}

At post offices, gas stations, convenience stores and roadside booths located along roads leading into the RZ, drivers paid S\$3 to buy a daily ``license'', or S\$60 for a monthly one, that permitted them to enter to a 6.2 km$^{2}$ Restricted Zone (RZ), Monday through Saturday morning \citep{WatsonHolland1978}. The ``license'' was a paper decal that went in the windshield. Wardens at 22 access points wrote down the plate numbers of vehicles lacking licenses. At first, only private vehicles with fewer than three passengers had to show licenses; taxis, commercial goods vehicles, public vehicles, motorcycles, carpools and buses were exempt, but taxis lost their exemption within three weeks. While the charging period was 7:30-9:30 AM originally, authorities noticed a surge of traffic just after 9:30 AM and extended charging to 10:15 AM starting August 1, 1975. The Park-and-Ride scheme was essentially shut down within a few months for lack of ridership, and the new bus services did not draw many customers. 

There were many changes to ALS over the years.\footnote{See Table 1 of \citet[p. 98]{PhangToh1997} for a table listing the nature and dates of modifications to ALS. } In 1976, the price of a license was increased to S\$4 and a double rate charged to registered company cars, because firms were able to deduct the ALS as a business expense. In 1977, charges for taxis were cut to S\$2, because it was difficult to find a taxi downtown. In 1980, the standard license rose again to S\$5 (S\$10 for company cars). Throughout the 1980s, the boundaries of the RZ were expanded to enclose new real estate developments. 1989 saw a package of reforms: first, carpools, goods vehicles and motorcycles lost their exempt status (motorcycles would pay only S\$2); second, ALS added an evening charging period. (Note that the evening period was not an outbound charge, and a single license served for travel in both periods of the same day.) The last major change occured in January 1994, when a S\$2 charge was added for entry from 10:15 AM to 4:30 PM \citep{PhangToh2004}. A vehicle entering the RZ in either or both peaks needed a Whole Day license (also good between the peaks), while a driver who entered only between the peaks only needed the Part Day license. 

\subsection{Results}

The immediate result of implementing ALS was a sharp fall in entries to the RZ. Between March and October of 1975, entries during the 7:30-10:15AM charging interval fell by 44\%---well beyond RTAC's desired 25-30\% reduction \citep{WatsonHolland1978}. Speed results appear in Table \ref{tab:speed-singapore}, although a lack of reliable measurements means that pre-charging speeds are estimates. Commute trips into the zone primarily switched to bus and carpool, but substantial rescheduling to earlier times of day was also observed. Travelers who had traversed the RZ in the morning en route to destinations outside tended to switch to a ring road. 

In spite of the substantial changes in flows and behavior, a household survey of travel times conducted just after the launch yielded disappointing results. There was almost no effect on traffic in the charge-free evening peak; and, due to the fall in speeds on the ring-road and mode shifting to slower modes such as bus and carpool, average journey times actually worsened in the short term. 

An important point about the launch of ALS is that the initial price of a license was not decided by detailed demand or traffic studies \citep{WatsonHolland1978}. Rather, authorities simply noted that the rate of entries to the RZ was about 25-30\% lower between the peaks, when traffic was considered acceptable, and estimated that a S\$3 charge would discourage about this proportion of trips. Since the actual result exceeded expectations so drastically, some observers---including \citet{Wilson1988a,McCarthyTay1993} and \citet{WatsonHolland1978}---have concluded that, at least initially, charges were set too high.


\begin{table}[ht]

\begin{tabular}{c>{\centering}p{3cm}>{\centering}p{3cm}}
 & before ALS\\
(kph estimated) & after ALS \\
(kph observed)\tabularnewline
\cline{2-3} 
Restricted Zone & 27 & 33\tabularnewline
\cline{2-3} 
inbound radials & 29 & 32\tabularnewline
\cline{2-3} 
outbound radials & 35 & 35\tabularnewline
\cline{2-3} 
ring road & 25 & 20\tabularnewline
\end{tabular}

\caption{Singapore speeds before and after Area License Scheme implementation \citep[p.10]{WatsonHolland1978} }
\label{tab:speed-singapore}
\end{table}

The 1989 reforms yielded the intended results. Between May 1989 and May 1990, traffic composition in the morning rebalanced as a rise and entries by car and taxi partly offset roughly 50\% falls in entries by truck and motorcycle. In the new evening charging period, entries fell by 54\% and exits by 34\% \citep[p. 19]{Gomez-Ibanez1994}. Also, although less data are available for the 1994 introduction of the Part-Day pricing, traffic and congestion fell in the periods just after the morning and just before the evening period (Phang et al., 1997).

\subsection{Finances}

ALS could be said to have the highest rate-of-return of any downtown pricing scheme. While the capital cost of implementation was S\$6.6 million, in fact 95 percent of that cost was sunk into the park-and-ride system \citet[p. 38]{WatsonHolland1978}. Initial revenues from license sales were S\$225,000 per month and operating costs were S\$50,000 per month. By 1993, annual revenues were S\$47 million, of which operating costs consumed 9 percent \citep{PhangToh2004}. In September 1998, when the scheme ended, annual revenues were about S\$100 million \citep{Chin2010}. (CITE THE OTHER SOURCE FOR REVENUES, NOT CHIN)