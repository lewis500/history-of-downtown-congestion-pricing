\section{Singapore Area License Scheme}\label{sec:als}

\subsection{Implementation}

In the early 1970's, Singapore took an interest in downtown pricing for the same reason Britain had ten years earlier: quick growth in car ownership. Between 1960 and 1970, the private vehicle population doubled as the country grew wealthier and built residential estates in its hinterlands, but the length of public roads rose only 35\% \citep{Santos2004}. Traffic models in the Singapore Concept Plan of 1971 predicted the imbalance of roadspace and vehicles use would only grow worse with time. The government responded forcefully; in the 1970's, it consolidated the island's myriad bus operators into an integrated company, embarked on a campaign of new expressways and arterials, drafted plans for a metro (though construction was delayed until 1982), impose prohibitively high taxes and fees on car ownership\footnote{See Appendix B of \citet{Gomez-Ibanez1994} for a list of Singapore's policies against car ownership.} and, in late 1973, convened officials from several ministries as the Road Transport Action Committee, tasked with coming up with solutions to downtown congestion. 

The Committee's first proposals included bus lanes as well as well as encouraging staggered work hours and carpooling \citep{Chin1998}. But its most historic output was a May 1974 Report, \citet{SRTAC1974}, which proposed four policies: (i) increases in CBD parking fees; (ii) new commuter bus services; (iii) a park-and-ride scheme; (iv) a simple DCP system called the ``Area License Scheme'' (ALS).\footnote{It is possible the idea for ALS came when Gabriel Roth (of the Smeed committee), who was then working in Singapore as a consultant for the World Bank, passed the World Bank's plan for road pricing in Caracas, \citet{Vorhees1973}, to members of the RTAC (CITE ROTH 1996).} ALS commenced operation on June 3, 1975 and continued, with constant refinement, until 1998.

\subsection{Design}

At shops and roadside booths, drivers paid S\$3 to buy a daily ``license'' (S\$60 for a monthly license) to enter a 6.2 km$^{2}$ Restricted Zone (RZ) on Monday through Saturday mornings \citep{WatsonHolland1978}. The ``license'' was a paper decal that went in the windshield. Wardens standing by the road at 22 access points to the RZ wrote down the plate numbers of vehicles lacking licenses. At first only private vehicles with fewer than three passengers had to pay, with taxis, commercial goods vehicles, public vehicles, motorcycles, carpools and buses exempt; but taxis lost their exemption within three weeks due to a surge in taxi traffic within the zone. While the charging period was 7:30-9:30 AM originally, authorities  extended charging to 10:15 AM on August 1, 1975. 

ALS was subject to constant refinement \citep{Gomez-Ibanez1994, PhangToh1997}. In 1976, the price rose to S\$4 and a double rate was charged to registered company cars, because firms could deduct the charge from taxes. In 1977, charges for taxis were cut to S\$2. In 1980, the price rose to S\$5 (S\$10 for company cars). Throughout the 1980s the boundaries of the RZ were slightly expanded to encompass new real estate developments. In 1989, carpools, goods vehicles and motorcycles lost their exempt status, and an evening charging period was added from 4:30 PM to 6:30 PM. Note that the evening period was not an outbound charge, and a single license purchased travel in both periods. In January 1994, a S\$2 ``Part Day'' license was added for entry from 10:15 AM to 4:30 PM. A vehicle entering the RZ in either or both peaks needed a Whole Day license (also good between the peaks), while a driver who entered only between the peaks only needed the Part Day license. Finally, in June 1995 the government implemented the Road Pricing Scheme (RPS) consisting of tolls on one expressway, where drivers without an ALS license had to buy a cheaper RPS license to drive from 7:30-8:30 AM \citep{PhangToh2004}. Throughout, a monthly license was always twenty times the cost of a daily one.

\subsection{Results}

The World Bank, interested in promoting DCP, conducted extensive studies of the scheme's original effects; the results appear as \citet{WatsonHolland1978} and are summarized, with much additional information, in \citet{Gomez-Ibanez1994}. 

Between March and October of 1975, vehicle entries to the RZ during the 7:30-10:15AM charging interval fell by 44\%---well beyond RTAC's desired 25-30\% reduction---due to a 73\% fall in entries by car. The (exempt) 4+ occupancy carpool share of traffic rose from 10\% to 44\%. Commute trips into the RZ primarily switched to bus and carpool, while travelers who had traversed the RZ in the morning to destinations outside tended to switch to a ring road. 

Effects on speed were somewhat less dramatic, and there was extensive additional congestion on the unpriced ring road around the RZ (see Table \ref{tab:speed-singapore}). Because of the ring road congestion, rerouting and shifting to slower modes, a survey of people's door-to-door travel times revealed a disappointing increase---at least initially. 

\begin{table}[ht]

\begin{tabular}{c>{\centering}p{3cm}>{\centering}p{3cm}}
 & before ALS\\
(kph estimated) & after ALS \\
(kph observed)\tabularnewline
\cline{2-3} 
Restricted Zone & 27 & 33\tabularnewline
\cline{2-3} 
inbound radials & 29 & 32\tabularnewline
\cline{2-3} 
outbound radials & 35 & 35\tabularnewline
\cline{2-3} 
ring road & 25 & 20\tabularnewline
\end{tabular}

\caption{Singapore speeds before and after Area License Scheme implementation \citep[p.10]{WatsonHolland1978} }
\label{tab:speed-singapore}
\end{table}

The scheme looks more effective in the long run, especially after the 1989 reforms. In 1991, speeds in the morning and evening charging periods were 33 kph and 32 kph, repectively, compared to just 19 kph in 1975 \citep{Menon1993}. Given the island's rising population and, especially, its tremendous economic growth from 1975-1991, this speed increase is remarkable. The 1994 introduction of the Part-Day license was also effective at rebalancing traffic \citep{PhangToh1997}, and the RPS led to a doubling of traffic speeds on the expressway where it was applied \citep{PhangToh2004}.

In line with Theme II, the Land Transport Authority of Singapore has cited the dramatic switch from private cars to carpool and bus that accompanied the introduction of ALS as an example of \citet{Shampanier2007}'s argument, noting ``what we now know about the zero-price effect gives us some hints that standard economics is not fully accounting for the strength of motorists’ aversion to the ALS'' \citep[p.17]{Lew2009}. Moreover, despite price increases in 1976 and 1980 and the introduction of a double rate for company cars in 1976, entries during charging hours rose every year from 1975-1982, whereafter a recession sharply reduced entries \citep[pp. 17-18, Tab. 7]{Gomez-Ibanez1994}. 

\subsection{Finances}

ALS was, proportionately, the most profitable DCP system. The capital cost of implementation was S\$6.6 million, but 95 percent of that expense was the park-and-ride system \citep[p. 38]{WatsonHolland1978}. The park-and-ride system was not essential; it was essentially shuttered within a few months of the launch for lack of ridership. Initial revenues from license sales were S\$225,000 per month and operating costs were S\$50,000 per month. In 1992, revenues were S\$38 million and operating costs about 9 percent \citep[[p. 21]{Gomez-Ibanez1994}. In 1998, (just before the scheme ended) ALS and RPS earned about S\$100 million \citep{Goh2002,Chin2009}.