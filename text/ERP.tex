\section{Singapore --- Electronic Road Pricing (ERP)}

\subsection{History}

Over time, ALS became hard to administer as Singaporean authorities tried to price vehicles more precisely. \citet[p. 4]{Chin2010} describes ALS once multiple charging periods and vehicle classes had been added: ``There were 16 different types of licences in use at its peak, and much concentration by the enforcement officers was required to ensure that they identified them correctly.'' The sticker system also precluded charging vehicles for each entry, and made it impossible to introduce more than two charging periods.

Therefore, in 1989 Singapore solicited bids for a wireless system which would be called, like Hong Kong's, Electronic Road Pricing (ERP). The government was wary of the privacy concerns that had helped sink Hong Kong's ERP \citep{PhangToh1997,Chin2010}. Thus, rather than a tag-and-beacon system, they chose a ``smart card'' system with a different direction of information flow: instead of having roadside beacons identify which cars pass, instead the beacon tells a device on the vehicle to deduct money from a stored-value car---similar to RFID cards used for transit today. In 1995, the contract for the system was awarded to a contract led by Phillips Singapore, and after extensive field trials and publicity, ERP commenced operation on September 1st, 1998.

\subsection{Design}

\citet{Menon2004} describes ERP as having three components.

\begin{itemize}
\item In-vehicle Unit (IU), an electronic device that sits on vehicles' dashboards or handlebars and. The IU has a slot for a ``CashCard'' that the user can load up with money at various places.\footnote{The CashCard can also be used to pay for parking. In March 2015, a sevice called vCashCard was announced that allows drivers to pay with debit or credit without having a physical CashCard in the IU. (http://www.straitstimes.com/singapore/transport/virtual-cashcard-aims-to-solve-erp-woes)} There are IU's for each of six vehicle classes (see Figure \ref{fig:singapore-IUs}), because the toll charged to a vehicle is a multiple of its ``Passenger Car Unit'' (PCU), an index of how much roadspace the vehicle takes up (see Table \ref{tab:passenger-car-units}). 

\item Gantries (see Fig. \ref{fig:singapore-gantry}). Gantries are positioned in pairs. When a vehicle passes below, the first gantry signals the vehicle's IU, telling it to charge the CashCard appropriately. An optical sensor on the second gantry confirms the vehicle type matches the IU. In the event of error, a camera on the first gantry captures the rear number plate. 

\item The central control system, which verifies charges and issues notices and fines when there is a violation or error.
\end{itemize}

\begin{table}
	\begin{tabular}{|c|c|}
		\hline 
		PCU's & Vehicles\tabularnewline                              
		\hline 
		\hline 
		0.5   & motorcycles\tabularnewline                           
		\hline 
		1     & cars, taxis, light goods vehicles\tabularnewline     
		\hline 
		1.5   & heavy goods vehicles/small buses\tabularnewline      
		\hline 
		2     & very heavy goods vehicles/large buses\tabularnewline 
		\hline 
	\end{tabular}
	
	\caption{
	Passenger Car Units (PCU's) for ERP. Vehicle charges are weighted by the PCU number---e.g., a very heavy goods vehicle pays four times what a motorcycle does. \citep{LTA2016} 
	}
	\label{tab:passenger-car-units}
\end{table}

ERP varies tolls substantially by time-of-day. Figure \ref{fig:singapore-toll-schedule} illustrates a weekday toll schedule for the CBD cordon. For the first five years of ERP, tolls changed only at half-hour intervals. But since February 2003, whenever tolls change by more than S\$1, there is a five-minute interval in which tolls rise or fall by half the amount of the change, in order to discourage cars from slowing down or speeding up when tolls are about to change \citep{Menon2004}. The Land Transport Authority (LTA) updates prices quarterly to maintain speeds of 45-65kph on expressways and 20-30 kph on roads in the RZ, because the LTA believes these speeds maximize flow \citep{Li1999}.
	
\begin{figure}
	\includegraphics[width=4in]{../img/singapore-IUs.jpg}
	\caption{ERP In-vehicle units \citep{LTA2016}}
	\label{fig:singapore-IUs}
\end{figure}

\begin{figure}
	\includegraphics[width=4in]{../img/singapore-gantry.jpeg}
	\caption{ERP gantry \citep{LTA2016}}
	\label{fig:singapore-gantry}
\end{figure}


\begin{figure}
	\includegraphics[width=1\textwidth]{../img/singapore-prices.png}
	\caption{Singapore prices for different years. The schedule becomes more variable as years pass. }
	\label{fig:singapore-toll-schedule}
\end{figure}

The substantial changes to ERP have been spatial. ERP started in 1998 with 33 gantries that approximately reproduced the ALS cordon. Beginning in 1999, the LTA added gantries gradually to enclose the first cordon in a second ``Outer Cordon'' \citep{Chin2010}. In 2005, a shopping area on the border of the CBD Cordon became a sub-cordon called the Orchard Road Cordon, and gantries were added to charge outbound trips in the evening on one expressway. In 2008 a line of gantries, operational from 6-8 PM was planted down the middle of the CBD Cordon. By December 2014, ERP made use of 80 tolling sites \citep[p. 406]{Chu2015}.

\subsection{Results}

The transition to ERP was not as thoroughly documented as the launch of ALS. One significant effect is that entries to the CBD fell by about 15\%, largely due to a decline in repeat trips by the same vehicle \citep{Menon2000}. Since ALS permitted unlimited same-day entries, under ALS about 23\% of trips had been repeat trips---e.g., office workers using cars for lunches and meetings in the middle of the day \citep[p. 23]{Chin2010}. \citet{Olszewski2005} conclude, using data from before and after ERP, that the LTA's charging structure has done a good job controlling congestion and spreading traffic flow over the peaks.

\subsection{Finances}

Implementation cost S\$197 million in 1998, of which S\$100 million paid for IU's (since the launch this expensive is incurred by the vehicle owner) and S\$97 million to build out the infrastructure \citep{Santos2004}. Annual operating costs have measured about 20-30 percent of revenues \citep{Chin2010}.

ERP revenues in 1999 were S\$68 million---down a third from the S\$100 million earned by ALS and the Road Pricing Scheme in 1998 \citep[p. 34]{Goh2002}. Revenues fell because ERP prices were lower than the ALS charge, which, combined with the significant investment cost and higher operating cost, makes it clear that the switch to ERP was not motivated by revenue concerns. Singapore does not regularly report revenues, but they were S\$159 million \citep{Chen2012}. Revenue is not hypothecated but rather flows into the governent's general fund.

