\section{The future}\label{sec:future}

The story of DCP is certainly unfinished. There are now several important developments underway around the world, including revamping of extant schemes and the possible implementation of new ones. One innovation that now seems to be on the table is distance-based charging.

The most likely source of major developments is Singapore. Singapore has already started the process of changing ERP into ERP 2.0, a distance-based charge, administered by satellite tracking and roadside cameras \citep{Straits2016}. ERP 2.0 will launch in 2020, and will also be used for on-street parking charges.

In London, the LCC infrastructure is being purposed for pollution control. On October 23, 2017, TfL commenced operation of the ``T-charge'' system. This scheme charges an additional \pounds 10 to older vehicles with higher emissions to enter the CZ during the same charging period as the LCC \citep{BBC2017}. The measure is a step on the way to the April 2019 implementation of an ``Ultra-Low Emission Zone,'' which would enact much higher charges (e.g., up to \pounds 100 for certain trucks) on a broader range of vehicles than the T-Charge (including motorcycles) and apply 24 hours every day of the year.\footnote{https://tfl.gov.uk/modes/driving/ultra-low-emission-zone} 

The Swedish Ministry of Finance recently proposed to increase the Stockholm Congestion Tax, by 10 SEK at the peaks (28\%), and to extend charging to more days of the year, including part of July \citep{Mitti2017}. The Stockholm government is receptive to the idea as long as revenues are invented in local transportation.

In Vancouver, local governments and the regional transportation authority (TransLink) have established a special commission tasked with researching options for congestion pricing. Their first report, \citep{vancouver2018}, considers---among more traditional scheme designs---implementing DCP using a distance-based charge. 

Finally, in fall of 2017, Andrew Cuomo, governor of the State of New York, appointed a task force called ``Fix NYC'' to look again at DCP for lower Manhattan (another DCP scheme was nearly passed in 2008). The impetus for the plan is a transit funding gap. In January 2018, FixNYC released its first report \citep{FixNYC2018} outlining possible scheme designs. The Report seems to favor a two-pronged approach. The first prong is  oa charge on ridesharing trips---including, perhaps, time- and distance-based charges. The second is a flat cordon toll, like Milan's, for entry by ordinary traffic during the workday. 

% In addition, 
