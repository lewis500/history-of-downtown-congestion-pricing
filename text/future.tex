\section{The future}\label{sec:future}

The story of DCP is certainly unfinished. There are now several important developments underway, including revamping of extant schemes and the possible implementation of new ones.

The most likely source of major developments is Singapore. Singapore has already started the process of changing ERP into ERP 2.0, a distance-based charge, administered by satellite tracking and roadside cameras.\footnote{http://www.straitstimes.com/singapore/transport/erp-20-goes-the-distance-with-new-tech} The new system is planned to launch in 2020, and will also incorporate parking charges.

In London, the LCC infrastructure is being purposed for pollution control. On October 23, 2017, TfL commenced operation of the ``T-charge'' system. This scheme charges an additional \pounds 10 to older vehicles with higher emissions to enter the CZ during the same charging period as the LCC.\footnote{ ``T-Charge: New London traffic charge comes into force'' http://www.bbc.com/news/uk-england-london-41695116} The measure is a step on the way to the April 2019 implementation of an ``Ultra-Low Emission Zone,'' which would enact much higher charges (e.g., up to \pounds 100 for certain trucks) on a broader range of vehicles than the T-Charge (including motorcycles) and apply 24 hours every day of the year.\footnote{``Ultra Low Emission Zone'' https://tfl.gov.uk/modes/driving/ultra-low-emission-zone} London Mayor Sadiq Kahn has also entertained the idea of replacing the LCC with a distance-based charge.\footnote{https://www.citylab.com/transportation/2017/06/london-driving-mileage-fee-sadiq-khan/531270/}

The Swedish Ministry of Finance recently proposed to increase the Stockholm Congestion Tax, by 10 SEK at the peaks (28\%), and to extend charging to more days of the year.\footnote{https://mitti.se/nyheter/trafik/forslaget-dyrare-trangselskatt/?omrade=hela-stockholm} The Stockholm government is receptive to the idea as long as revenues are invented in local transportation.

In Vancouver, local governments and the regional transportation authority (TransLink) have established a special commission tasked with researching options for congestion pricing. Their first report, \citep{vancouver2018}, considers---among more traditional scheme designs---implementing DCP using a distance-based charge. 

In Fall of 2017, Andrew Cuomo, governor of the state of New York, appointed a task force called FixNYC to reexamine at DCP for lower Manhattan.\footnote{https://www.nytimes.com/2018/01/16/nyregion/cuomos-congestion-pricing-for-new-york-city-begins-to-take-shape.html} The task force has yet to release a report, but it reportedly will incorporate a cordon toll as well as distance- and time-based levies on taxis and ridesharing vehicles.


% ERP 2.0

% The San Francisco County Transportation Authority is currently running a long-term study 

% SAN FRANSCISCO MAPS STUDY. NEW YORK CITY. VANCOUVER.
