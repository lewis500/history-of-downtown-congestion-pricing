\section{Discussion}\label{sec:discussion}

This chapter has reviewed the history of zone pricing from its conception as an idea in the 1950's and 1960's to its proliferation in the new millenium as a result of number-plate recognition technology. Surveying the zone pricing experience, two themes emerge that are worth highlighting.

First, note that all the systems surveyed produced similar traffic reductions, between 10 and 20 percent of all entries. This is surprising given that the amount of money charged varies considerably. In Stockholm, the maximum charge is only about \$3. In London it was about \$18 before the recent devaluation of the pound. Moreoever, the models in Stockholm and Gothenburg were mistaken in that they predicted different traffic reductions in the peak and off-peak, due to the time-varying toll. Instead what was observed is that a similar share of traffic stopped driving at both times of day. As a rule-of-thumb, it might be worthwhile to assume that, in any pre-charging population of drivers, about 10-15\% of drivers are simply unwilling to pay anything to drive, almost regardless of the size of the toll.

Second, while most of the theory of congestion pricing focuses on prices, in practice exemptions are critical. Some of these exemptions---such as those for the handicapped or medical vehicles---are easy to justify by appeals to welfare or social justice. But many exemptions---such as those for taxis or even, in the case of Milan, refrigerated delivery trucks---lack much of a rationale outside politics.