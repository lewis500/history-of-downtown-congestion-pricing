\section{Conclusion}\label{sec:conclusion}

This paper has reviewed the history of DCP, from its first appearance in the minds of British and American academics in the 1950's to current plans for new and reconfigured systems across the world. Moreover, even among the small number of schemes implemented so far, there is enormous diversity in implementation, design, effects and finances. But experience shows that DCP is effective at reducing entries to tolled zones by charged  trips, raising speeds, raising funds and encouraging people to travel by priveleged modes and vehicles. By way of a conclusion, we now briefly mention some developments omitted for brevity or want of literature and return to the themes proposed in Sec. \ref{ssec:themes}.

\subsection{Themes}

Theme I: exemptions are surprisingly consequential. The ALS, LCC and SCT all saw taxi traffic jump upon implementation. The LCC charging zone has been swamped by exempt private-hire vehicles in recent years, while for the SCT exempt green vehicles rose to 14\% of traffic before having their exemption cancelled. In Gothenburg, use of the ``multi-passage'' rule was underpredicted. In Milan, the chargable share of traffic fell sharply during the Ecopass regime, and exempt traffic rose quickly, while Area C seems to have encouraged the use of motorcycles. The apparent lesson of Theme I is that cities should be cautious about handing out exemptions. Rather than exempting green vehicles, for instance, a city might do better to ban or charge high-emission vehicles---as London and Milan do explicitly and Singapore does indirectly (by charging larger vehicles more).

Theme II: the fact of a toll is much more consequential than differences in the toll level. The ALS, LCC, SCT and GCT all demonstrated little change in traffic after increasing tolls but dramatic reductions upon introducing tolls, while ERP seems to be an exception and Milan has not increased tolls yet for a particular vehicle class. If true, Theme II has major rammifications for policy. If almost \emph{any} non-zero toll really does dissuade so many drivers, then a city where fairness issues loom large could simply charge low tolls. In particular, if speed in some city responds non-linearly to traffic in-flows, then even a low toll could garner much of the traffic gains of an ``efficient'' toll while imposing less hardship. Of course, choosing low tolls has downsides: the scheme will raise less money and---since large-enough toll increases certainly discourage \emph{some} traffic---permit more congestion.
