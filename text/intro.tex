\section{Introduction}

INTRO PARAGRAPH

This paper provides a history of downtown road pricing. 



\subsection{Goals and Scope}

The goals of this paper are to (i) summarize the history of DRP up to 2017; and (ii) briefly describe the schemes that have been implemented; (iii) point interest researchers to cited works where they can find more detailed information. Each scheme is treated in four sections: (i) implementation, (ii) design, (iii) results and (iv) finances. However, relative to the works dedicated to particular scheme, there is less attention paid to the scheme's impacts and more to the story of how each scheme came to be.

To save space and keep the narrative focused, the scope is limited. Detail is only provided for systems that are (i) implemented; (ii) reasonably large; and (iii) designed to reduce congestion. Sec XXX (``Transition Period'') mentions Hong Kong's Electronic Road Pricing and the Cambridge congestion metering system, which were trialed but never implemented, as well as Norway's toll rings, which are not congestion relief measures. Requirement (ii) excludes the very small Durham and Valetta measures, which are idiosyncratic and about which not as much information is available. In Section XXX we mention cities where DRP was only proposed but never trialed.

In addition, a link is provided to a list of valuable sources on DRP, for researchers wanting to delve more deeply.

\subsection{Types of system}

% Two pricing designs are relevant: 

% \begin{enumerate}
% \item \emph{Tag-and-beacon}: A roadside ``interrogator'' emits electromagnetic waves on some spectrum. Vehicles carry transponders which, when struck by the interrogator's energy, turn on and return a signal with a key that identifies the vehicle or an account to charge. \footnote{Essentially, this is the same process by which human vision works: when photons strike matter, its atoms become energized and emit photons. The difference is that tag-and-beacon systems operate on an invisible part of the spectrum where waves came pass through most solid objects.} As far as information flow, what distinguishes this system is a \emph{two-way} flow wherein the vehicle identifies itself to a central system.
% \item \emph{}
% \item \emph{In-vehicle units}: Vehicles come equipped with some 
% \end{enumerate}

% This scope excludes much valuable information about other schemes that may interest scholars. Requirement (i) excludes Hong Kong's ERP and Cambridge's congestion metering system, which were both trialed but never implemented. Requirement (ii) excludes very small systems such as Durham's and Valetta's. Requirement (iii) excludes the Norwegian toll rings (see \citet{Ieromonachou2006} and \citet{Ramjerdi2004}).
