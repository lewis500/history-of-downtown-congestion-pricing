\section{Introduction}

Today, authorities in Vancouver and New York City are considering road tolls as a means of relieving traffic congestion on city streets in central areas---a practice this paper calls, hereafter, ``downtown congestion pricing'' (DCP). Forms of congestion pricing are already in place on highways and bridges throughout North America, such as Highway 407 in Toronto and the High-Occupancy Toll (or ``Express'') Lanes now proliferating across the United States. But the schemes being considered for New York City and Vancouver would differ from not only by their setting but also topologically: while congestion pricing on highways charges drivers to travel between well-defined start and end points, DCP charges for use of  clusters of contiguous downtown streets (``zones''); charged trips can thus begin and end at nearly infinite points within the zone or along its boundary. As an analogy, highway pricing resembles a theme park where customers buy tickets for particular rides, whereas DCP is more similar to a park where customers pay to enter. Thus, DCP is attended by administrative and technological concerns that warrant special treatment.

To date, only five cities have implented DCP as we have defined it: Singapore, London, Stockholm, Gothenburg and Milan. This paper provides a broad sweep of their experiences along with some historical connective tissue. The goal is to help scholars working on issues related to DCP to quickly become familiar with what has already happened and to point towards cited works containing further information. Note there are already several papers with similar aims. In particular, \citet{Hau1992}, \citet{Gomez-Ibanez1994}, \citet{Small1998}, \citet{Walker2011} and \citet{Anas2011} survey developments up to their publications, with \citet{Gomez-Ibanez1994} being especially thorough. 

\subsection{Organization}

There are sections devoted to six DCP schemes: Singapore's Area License Scheme (\ref{sec:als}), Singapore's Electronic Road Pricing (Sec. \ref{sec:erp}), the London Congestion Charge (Sec. \ref{sec:london}), the Stockholm Congestion Tax (Sec. \ref{sec:stockholm}), Milan's Ecopass and Area C (Sec. \ref{sec:milan}) and the Gothenburg Congestion Tax (Sec. \ref{sec:gothenburg}). Each section has four subsections:

\begin{enumerate}
    \item \emph{Background}. Motivation and events leading up to implementation.
    \item \emph{Design}. The scheme's technology, prices, exemptions and geography---including changes over time.
    \item \emph{Transportation Impacts}. The impact on travel---especially the effect on the magnitude and composition of flow into the covered zone and, where available, on traffic speed and transit ridership. Note that DCP also affects air pollution, equity, real estate and commerce. But the transportation impacts are the most well documented, comparable and, arguably, \emph{direct} impacts. The reader interested in other effects---which are undoubtedly important---can always find such information in the works cited.
    \item \emph{Finances}. How much money the scheme has cost to build and operate, and how much revenue it has raised.
\end{enumerate}

In between the sections for specific schemes are sections devoted to historical context. Sec. \ref{sec:early} describes intelletual developments prior to real implementation. Sec. \ref{sec:transition} describes a twenty-year ``transition period'' following Singapore's first scheme. Sec. \ref{sec:future} briefly describes prospects for future developments. 

\subsection{Themes}\label{ssec:themes}

The paper is more focused on description than inference, but we do point out two recurring themes.

Theme I is that exemptions are among the most consequential aspects of a scheme's design, to a degree that often surprises authorities. Exempt classes of trips or vehicles quickly become a large share of traffic, and in doing so they compromise congestion goals and reduce revenue more than projected.

Theme II is that charging implementing a toll is much more consequential than differences in the toll level. When a scheme is implemented, entries to the tolled zone by vehicles subject to charge drop by 2\% or more, but later toll increases---even drastic ones---hardly discourage driving at all. Noting this effect in Stockholm and Gothenburg, \citet[p. 45]{Borjesson2018} offer two explanations. The first is grounded in traditional economic theory: the initial toll discourages all the drivers with elastic demands, so the traffic exposed to later toll increases is composed of drivers with a high willingness-to-pay for travel. The second explanation is grounded in behavioral economics: \citet{Shampanier2007} note that consumers are extremely sensitive to being charged \emph{any} positive price---even negligible ones.



