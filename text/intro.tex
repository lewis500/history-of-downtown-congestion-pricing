\section{Introduction}

MENTION THE MOVE NY PROJECT

In the past twenty years, congestion pricing has been implemented in various forms. Especially popular in the United States are variable tolls on certain freeway lanes (CITE EXAMPLE). But the New York proposal is different: drivers would pay tolls not to access particular lanes or roads but rather to access a busy \emph{area} of a city.  The literature on congestion pricing distinguishes several forms of such scheme---e.g., a ``cordon toll'' charges drivers to cross a boundary around the charged zone, while ``metering'' charges them for how long or how far they travel inside the zone. But what unites such schemes is that, for purposes of charging, contiguous sets of a downtown street network are treated as singular entities. Hence, hereafter we will call this practice ``downtown congestion pricing'' (DCP). An analogy to theme parks is helpful: in some theme parks one pays a ticket or money for each ride, while in others one buys an all-day ticket. Road tolls are like the former, DCP like the latter.

% This paper reviews the experience of DCP thus far. The goals are to brief the reader on what has happened and to point toward more detailed works. To save space,  detailed discussion is restricted to systems that are (i) actually implemented; (ii) reasonably large; and (iii) designed to reduce congestion.\footnote{Requirement (i) eliminates merely many proposed schemes, such those in Cambridge, New York City and Edinburgh, although we mention Hong Kong's ERP to provide context. Requirement (ii) excludes the very small Durham and Valetta measures. Requirement (iii) excludes schemes like the Norwegian toll rings although these are mentioned again for context. 
% } These are Singapore's Area License Scheme and Electronic Road Pricing, the London Congestion Charge, the Stockholm and Gothenburg Congestion Taxes and Milan's Ecopass and Area C schemes. Each scheme is treated in four sections: (i) implementation, (ii) design, (iii) results and (iv) finances.

For every DCP scheme in existence there is already a host of scholarly material. GIVE EXAMPLES. However, there is as yet no single document which gives, at one pass, an overview of all the schemes implemented thus far. This paper is intended to do so, so that the reader interested in proposals like New York's can quickly learn the facts of what has happened and find references to more detailed works. BLAH BLAH.

The chief aim of this paper is to review, not to compare and contrast schemes. Nonetheless,it will be obvious enough that three motiffs emerge: 

\begin{enumerate}
    \item The costs of exemptions and special treatment.
    \item Constant change, not only to toll rates but also to fundamental features of the schemes.
    \item A crisis. 
\end{enumerate}

% Relative to the specialized works, relatively more space is reserved for explaining how the schemes came to be and how they operate, rather than what their effects are. These are the less disputable aspects of our knowledge, and pricing has so many measurable effects that any document trying to list more than a few of them would require a book.

% In the course of the history, a few themes emerge. First, exemptions are among the most important features of a scheme, but they are usually decided without much thought. Second, ANPR is what has allowed DCP to proliferate. 