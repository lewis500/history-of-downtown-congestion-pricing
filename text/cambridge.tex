
% \subsection{Cambridge, UK congestion metering}

% In 1990 the Cambridgeshire County Council adopted a Cambridge Transport Strategy (CTS) consisting of a package of policies to deal with growing congestion, including a new light rail line through the city \citep{Ison1996}. Later in the year, the Council, faced with a \pounds 20 million gap in funding for the light rail, gave preliminary approval to studying road pricing---partly as a way to  pay for the light rail. The idea was heavily pushed by Brian Oldridge, the County's Director of Transportation \citep[p.55-57]{Gomez-Ibanez1994}. 

% The system envisioned for Cambridge was called ``congestion metering'' \citep{Ison1996}. Every vehicle near Cambridge would have an in-vehicle unit (IU) connected to its odometer. The IU would have a mouth for a ``smart card'' that drivers could top up with value---like the RFID touch cards now used for transit (e.g., the Oyster Card in London). When the vehicle crossed a cordon around Cambridge in the inbound direction at certain times-of-day, roadside beacons would turn it on; but, once activated, the IU would not begin to draw down the smart card's balance unless measurements suggested the car had become stuck in traffic---e.g., if it took three minutes to move a half kilometer. The idea was to charge drivers only when they create congestion.

% In October 1993, field trials were conducted on the technology as part of the ADEPT project at Newcastle-upon-Tyne University \citep{Blythe1993}. These trials had equipped cars drive between two beacons placed along a single road, and they suggested the technology was feasible. But ultimately nothing came of the plans: Oldrige's retirement, a change of local government and negative public opinion ended the effort \citep{Ison1998}.